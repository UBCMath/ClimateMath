
\documentclass[12pt]{article}
%%%%%%%%%%%%%%%%%%%%%%%%%%%%%%%%%%%%%%%%%%%%%%%%%%%%%%%%%%%%%%%%%%%%%%%%%%%%%%%%%%%%%%%%%%%%%%%%%%%%%%%%%%%%%%%%%%%%%%%%%%%%
\usepackage{amsmath}
\usepackage{graphicx}

%TCIDATA{OutputFilter=LATEX.DLL}
%TCIDATA{LastRevised=Friday, October 21, 2022 15:44:44}
%TCIDATA{<META NAME="GraphicsSave" CONTENT="32">}
%TCIDATA{Language=American English}

\addtolength{\topmargin}{-30mm}
\addtolength{\oddsidemargin}{-10mm}
\addtolength{\textwidth}{20mm}
\addtolength{\textheight}{65mm}
\setlength{\parindent}{0mm}
%\input{tcilatex}

\begin{document}


\pagestyle{empty}

\begin{center}
{\Large MATH 257/316 Class Project}\textbf{\ \\[0pt]
}



\textbf{A model to investigate the parameters that determine the possible
extinction of a fish population}

Anthony Pierce and Peter Harrington
\end{center}


Many species are constrained within a habitat of a given size. The size of the habitat may be limited due to the environment (e.g. there are only certain temperatures a species can tolerate, competition with other species, prey availability, or human induced mortality). A classic problem in mathematical ecology is to determine the minimum habitat size required for a species with a given growth rate to survive (Kot 2001).


The Fisher equation (\ref{Fisher_Unscaled}) is one model for the growth and dispersal of a species within it's habitat. Here let us consider a fish population $P(X,T)$ within a marine reserve of length $L$. Intense fishing occurs outside of the reserve, so the fish population can only survive in the habitat $X\in (0,L)$. The boundary conditions $P(0,T)=0$ and $P(L,T)=0$ represent the mortality due to fishing outside of the reserve. Here we are assuming the fishing fleet is very efficient and removes all fish that reach $X=0$ and $X=L$.


\begin{eqnarray}
\frac{\partial P}{\partial T} &=&D\frac{\partial ^{2}P}{\partial X^{2}}%
+\gamma P(1-\frac{P}{P_c}),~~0<X<L,~~T>0  \label{Fisher_Unscaled} \\
BC &:&P(0,T)=0,\text{ \ }P(L,T)=0  \notag \\
IC &:&P(X,0)=F(X)  \notag
\end{eqnarray}%
Here $D$ represents the rate at which the fish disperse within the marine reserve, $\gamma $ represents the birth rate of the fish at low population densities, $P_c$ represents the carrying capacity of the environment, and $F(X)$ represents the fish population sampled at some initial time.

(a) By introducing the scaled variables $x=X/L,$ $t=T/T_{0},$ and $%
u=P/P_c, $ reduce the boundary value problem (\ref{Fisher_Unscaled}) to
the following dimensionless form: 
\begin{eqnarray}
u_{t} &=&\alpha ^{2}u_{xx}+u(1-u),~~0<x<1,~~t>0  \label{Fisher} \\
BC &:&u(0,t)=0,\text{ \ }u(1,t)=0  \notag \\
IC &:&u(x,0)=f(x),  \notag
\end{eqnarray}%
where the dimensionless diffusion coefficient $\alpha ^{2}$ controls the
evolution of the fishery.

(b) Now explore the possibility of extinction of the fish population $%
u\equiv 0$ by considering whether a small perturbation $\widetilde{u}$ to
the zero solution, i.e., $u=0+$ $\widetilde{u},$ will grow or decay. By
substituting this perturbation into (\ref{Fisher}) and retaining only first
order terms derive the linearized Fisher equation:%
\begin{eqnarray}
\tilde{u}_{t} &=&\alpha ^{2}\tilde{u}_{xx}+\tilde{u},~~0<x<1,~~t>0
\label{Linear_Fisher} \\
BC &:&\tilde{u}_{x}(0,t)=0,\text{ \ }\tilde{u}(1,t)=0  \notag \\
IC &:&\tilde{u}(x,0)=\widetilde{f}(x)  \notag
\end{eqnarray}%
(c) Use the method of separation of variables to solve the above boundary
value problem (\ref{Linear_Fisher}) for $\tilde{u}(x,t)$.\bigskip \bigskip

\begin{center}
(Continued on the next page)\pagebreak
\end{center}

(d) Identify a condition on the dimensionless parameter $\alpha ^{2}$ that
characterizes the boundary $\alpha _{c}^{2}$ between extinction and
persistence of the fish population. Interpret your results to determine the
minimal length of the marine reserve that will ensure persistence of the
fish population as a function of the other parameters in the form $%
L=g(D,\gamma ).$

(e) Modify the MATLAB code provided in lecture 8 to solve the fully
nonlinear Fisher equation (\ref{Fisher}) using finite differences. Now
explore the sharpness of the extinction/persistence boundary by observing
the solution for $\alpha ^{2}=0.4$ and $\alpha ^{2}=0.41$ larger than and
just smaller than $\alpha _{c}^{2}.$ Use an initial perturbation 
\begin{equation*}
u(x,0)=0.1e^{(-64(x-\frac{1}{2})^{2})}
\end{equation*}%
Integrate the solution till $t=800$, determine $u(0,800),$\ and plot $%
u(x,t=800)$ in both cases $\alpha ^{2}=0.4$ and $\alpha ^{2}=0.41$. For
stability adjust the parameter $Nt$ to $Nt=2e6;.$


\vskip 1cm

\textbf{Interesting related references:}

\

H. Kierstead and L.B. Slobodkin. The size of water masses containing plankton blooms. \textit{Journalof Marine Research},12(1):141-147,1953.

\

John G. Skellam. Random dispersal in theoretical populations. \textit{Biometrika},38(1/2):196-218,1951.

\

Mark Kot. Elements of Mathematical Ecology, Chapters 15-17, Cambridge University Press, 2001 (Free online through UBC library)

\

Michael G. Neubert. Marine reserves and optimal harvesting, \textit{Ecology Letters}, 6:843-849, 2003.


\end{document}
